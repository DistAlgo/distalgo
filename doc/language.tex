\documentclass[12pt]{article}
\usepackage{fullpage}
\setlength{\parskip}{.8ex}
\setlength{\parindent}{0ex}
\usepackage{url}
\usepackage{xspace} %
\usepackage{alltt} %
\usepackage[T1]{fontenc} %
\usepackage{graphicx}

\usepackage{tikz}                %
\usetikzlibrary{arrows,positioning} 
\tikzset{
    %
    >=stealth',
    %
    absbox/.style={
           rectangle, rounded corners, draw=black, very thick,
           text width=14em, minimum height=2em,
           text centered},
    %
    absarrow/.style={
           <-, thick,
           shorten <=2pt,
           shorten >=2pt},
    field/.style={ %
           rectangle, draw=black, thick,
           text width=4.2em, minimum height=1.6em,
           text centered, inner sep=0pt},
    cell/.style={ %
           rectangle, draw=black, thick,
           text width=1.5em, minimum height=1.5em,
           text centered, inner sep=0pt},
    pointer/.style={ %
           <-, thick,
           shorten <=2pt,
           shorten >=2pt},
    vertex/.style={ %
           circle, draw=black, thick,
           text width=1.5em, minimum height=1.5em,
           text centered, inner sep=0pt},
}

\newcommand{\notes}[1]{} %
\newcommand{\todo}[1]{}
 %
\newcommand{\body}[1]{} %

\newcommand{\mysubsec}[1]{\subsection*{#1}}
\newcommand{\myparag}[1]{\vspace{2ex}\noindent\b{#1.}~}

 %

 %
 %





\hyphenation{in-cre-men-tal-i-za-tion di-men-sion-al}\todo{}
\def\hyph{-\penalty0\hskip0pt\relax}

\newcommand{\defn}[1]{{\it #1}\index{#1|bold}}
 %
 %
 %


 %
 %

\newcommand{\lang}[1]{#1\index{#1}}

\def\mathify#1{\ifmmode{#1}\else\mbox{$#1$}\fi}
\newcommand{\m}[1]{$#1$} %

\newenvironment{code}{\begin{center}\small\begin{alltt}\vspace{-1.5ex}}{\vspace{-1.5ex}\end{alltt}\end{center}}
\renewcommand{\c}[1]{{\tt\small #1}} %
 %
\renewcommand{\b}[1]{{\bf #1}} %





 %
 %
 %
 %
 %
 %
 %
 %
 %
 %
 %
 %
 %
 %
 %
 %
 %
 %
\newcommand{\p}[1]{{\small\mathify{\it #1}}}
 %
 %
\renewcommand{\to}{\c{->}\xspace} %
\renewcommand{\|}{\mathify{\,\mbox{\c{|}}\,}} %

 %
\renewcommand{\=}{\mathify{\,=\,}\linebreak[0]}




\newcommand{\s}[1]{\mathify{\#}\c{#1}}





 %
 %






 %

 %
 %




 %






 %
 %
 %
 %
 %


















\begin{document}

\body{} %

\newpage
\appendix

\renewcommand{\thesection}{\arabic{section}}

\setcounter{page}{1}
\setcounter{section}{0}


\notes{}

\section*{{\Large DistAlgo Language Description}}

{\large Yanhong A.\ Liu, Bo Lin, and Scott Stoller}

\c{liu@cs.stonybrook.edu, bolin@cs.stonybrook.edu, stoller@cs.stonybrook.edu}

{\small Revised January 16, 2017}
\vspace{4ex}

\def\da{\vspace{-2.5ex}in DistAlgo\vspace{-2.5ex}}
\def\langmap{\vspace{.5ex}\m{\longrightarrow}~\b{in Python syntax:~}
  \vspace{.0ex}}
 %
\renewcommand{\mysubsec}[1]{\section{#1}}
\renewcommand{\myparag}[1]{\subsection{#1}}


DistAlgo is a language for distributed algorithms.  We describe DistAlgo
language constructs as extensions to conventional object-oriented
programming languages, including a syntax for extensions to Python.

There are four components conceptually: (1) distributed processes and
sending messages, (2) control flows and receiving messages, (3) high-level
queries of message histories, and (4) configurations.

High-level queries are not specific to distributed algorithms, but using
them over message histories is particularly helpful for expressing and
understanding distributed algorithms at a high level.  Some conventional
programming languages, such as Python, support high-level queries to some
extent, but DistAlgo query constructs are more declarative, especially with
the support of tuple patterns for messages.

\mysubsec{Distributed processes and sending messages}

\myparag{Process definition}

A process definition is of the following form.  It defines a type of
processes named \p{p}, by defining a class \p{p} that extends class
\c{process}.  The \c{\p{process\_body}} is a set of method definitions and
handler definitions, to be described.
\begin{code}
    class \p{p} extends process:
      \p{process\_body}
\end{code}
The syntax of process definition could be made simpler and clearer:
\begin{code}
    process \p{p}:
      \p{process_body}
\end{code}
but it would make \c{process} a keyword, which is usually a reserved word,
whereas \c{process} as a class name is not reserved %
and can be defined or redefined to be anything else.

\langmap
\begin{code}
    class \p{p} (process):
      \p{process_body}
\end{code}

\todo{}


A special method \c{setup} may be defined in \c{\p{process\_body}} for
initially setting up data in the process before the process's execution
starts.
For each parameter \p{v} of \c{setup}, a process field named \p{v} is
defined automatically and assigned the value of parameter \p{v}; additional
fields
can be defined explicitly in the method body of \c{setup}.

A special method \c{run()}
may be defined in \c{\p{process\_body}} for carrying out the main flow of
execution.


A special variable \c{self} refers to the current process.  
Fields of the process may be defined by including the field name
as a parameter of method \c{setup}, or by explicitly prefixing the field name 
with \c{self} in an assignment to the field.
References to fields of the process 
do not need to be prefixed with \c{self}.  References to methods of the
process do not need to be prefixed with \c{self} either.  Also, method
definitions implicitly include parameter \c{self}.




\myparag{Process creation}

Process creation consists of statements for creating, setting up, and
starting processes. %

A process creation statement is of the following form.  It creates \p{n}
new processes of type \p{p} at each node in the value of \c{\p{node\_exp}},
and assigns the resulting process
or set of processes to variable \p{v}. %
Expression \c{\p{node\_exp}} evaluates to a node
or a set of nodes, specifying where the new processes will be created.
A node is a running DistAlgo program on a machine. %
A node is identified by a string of the form \c{name@host}, where \c{name}
can be specified on the command line when starting the node, and \c{host}
is the host name of the machine running the node; \c{@host} can be omitted
if the node is running on the same machine. %
All nodes communicating with each other must have the same cookie, which
can be specified on the command line when starting the node.
The number \p{n} and the \c{at} clause are optional; the defaults are 1 and
local node, respectively.  When both are omitted, a single process is
created and assigned to \p{v}; otherwise, a set of processes is created and
assigned to \p{v}.
\begin{code}
    \p{v} = \p{n} new \p{p} at \p{node_exp}
\end{code}

\langmap
\begin{code}
    \p{v} = new(\p{p}, num = \p{n}, at = \p{node_exp})
\end{code}



A process setup statement is of the following form.  It sets up the process
or set of processes that is the value of expression \p{pexp}, using method
\c{setup} of the process or processes with the values of argument
expressions \p{args}.  If the values of \p{args} are available when the
process or processes are created at a call to \c{new}, the call to
\c{setup} can be omitted by inserting tuple \c{(\p{args})} after \p{p} in
the call to \c{new}.
\begin{code}
    \p{pexp}.setup(\p{args})
\end{code}

\langmap%
\begin{code}
    setup(\p{pexp}, (\p{args}))        \normalfont{Note: You must add a trailing comma if \p{args} is a single argument.}
\end{code}

A process start statement is of the
 following form.  It starts the
execution of the method \c{run} of the process or set of processes that is
the value of expression \p{pexp}.
\begin{code}
    \p{pexp}.start()
\end{code}

\langmap%
\begin{code}
    start(\p{pexp})
\end{code}

\notes{}

\myparag{Sending messages}

A statement for sending messages is of the following form.  It sends the
message that is value of expression \p{mexp} to the process or set of
processes that is the value of expression \p{pexp}.  A message can be any
value but is by convention a tuple whose first component is a string,
called a tag, indicating the kind of the message.
\begin{code}
    send \p{mexp} to \p{pexp}
\end{code}

\langmap
\begin{code}
    send(\p{mexp}, to = \p{pexp})
\end{code}
\notes{}


\mysubsec{Control flows and receiving messages}

\myparag{Yield points}

A yield point preceding a statement is of the following form, where
identifier \p{l} is a label and is optional.  It specifies that point in
the program as a place where control yields to handling of un-handled
messages, if any, and resumes afterwards.
\begin{code}
    -- \p{l}:
\end{code}

\langmap
\begin{code}
    -- \p{l}
\end{code}
\quad\quad which is a statement in Python, where \p{l} is any valid Python
identifier.
\notes{}

\notes{}


\myparag{Handling messages received}

Handling messages received can be done using handler definitions and
message history variables.

A handler definition, also called a \c{receive} definition, is of the
following form.
It handles, at yield points labeled \p{l\sb{1}}, ..., \p{l\sb{j}},
un-handled messages that match \p{mexp} sent from \p{pexp}, where
\p{mexp} and \p{pexp} are parts of a pattern;
previously unbound variables in a pattern are bound to the corresponding
components in the value matched.
The \c{from} and \c{at} clauses are optional; the defaults are any process
and all yield points.  The \c{\p{handler\_body}} is a sequence of
statements to be executed for the matched messages.
\begin{code}
    receive \p{mexp} from \p{pexp} at \p{l\sb{1}}, ..., \p{l\sb{j}}:
      \p{handler_body}
\end{code}

\langmap\newpage
\begin{code}
    def receive(msg = \p{mexp}, from_ = \p{pexp}, at = (\p{l\sb{1}}, ..., \p{l\sb{j}})):
      \p{handler_body}
\end{code}
\quad\quad where \c{\_} is added after \c{from} because \c{from} is a
reserved word in Python.

Message histories, i.e., the sequences of messages received and sent, in
variables \c{received}
and \c{sent}, respectively, can be used in expressions.
Sequence \c{received} is updated at the next yield point if there are
un-handled messages, by adding un-handled messages before any matching
receive handler executes.
Sequence \c{sent} is updated at each \c{send} statement, by adding each
message sent to a process.

In particular, the following two equivalent expressions return true iff a
message that matches \p{mexp} sent from \p{pexp} is in \c{received}.  The
\c{from} clause is optional; the default is any process.
\begin{code}
    received \p{mexp} from \p{pexp}
    \p{mexp} from \p{pexp} in received
\end{code}

\langmap
\begin{code}
    received(\p{mexp}, from_ = \p{pexp})
    (\p{mexp}, \p{pexp}) in received
\end{code}

Similarly, the following expressions use \c{sent}.
\begin{code}
    sent \p{mexp} to \p{pexp}
    \p{mexp} to \p{pexp} in sent
\end{code}

\langmap
\begin{code}
    sent(\p{mexp}, to = \p{pexp})
    (\p{mexp}, \p{pexp}) in sent
\end{code}
\todo{}


\myparag{Synchronization}

Synchronization and associated actions can be expressed using general,
non\-deterministic \c{await} statements.

A simple \c{await} statement is of the following form.  It waits for the
value of Boolean-valued expression \p{bexp} to become true.
\begin{code}
    await \p{bexp}
\end{code}

\langmap
\begin{code}
    await(\p{bexp})
\end{code}

A general, nondeterministic \c{await} statement is of the following form.
It waits for any of the values of expressions \p{bexp\sb{1}}, ...,
\p{bexp\sb{k}} to become true or a timeout after \p{t} seconds, 
and then nondeterministically selects one\notes{} of statements
\p{stmt\sb{1}}, ..., \p{stmt\sb{k}}, \p{stmt} whose corresponding
conditions are satisfied to execute.  The \c{or} and \c{timeout} clauses
are optional.
\begin{code}
    await \p{bexp\sb{1}}: \p{stmt\sb{1}}
    or ... 
    or \p{bexp\sb{k}}: \p{stmt\sb{k}}
    timeout \p{t}: \p{stmt}
\end{code}
\notes{}

\langmap
\begin{code}
    if await(\p{bexp\sb{1}}): \p{stmt\sb{1}}
    elif ...
    elif \p{bexp\sb{k}}: \p{stmt\sb{k}}
    elif timeout(\p{t}): \p{stmt}
\end{code}

An \c{await} statement must be preceded by a yield point, for handling
messages while waiting; if a yield point is not specified explicitly, the
default is that all message handlers can be executed at this point.

\notes{}


\mysubsec{High-level queries of message histories}

\myparag{Comprehensions}

A comprehension is a query of the following form, plus a set of
\defn{parameters}---variables whose values are bound before the query.
For a query to be well-formed, every variable in it must be
\defn{reachable} from a parameter---be a parameter or be the left-side
variable of a membership clause whose right-side variables are reachable.
Given values of parameters, the query returns the set of values of \p{exp}
for all combinations of values of variables that satisfy all membership
clauses \c{\p{v_i} in \p{sexp\sb{i}}} and condition \p{bexp}.
When \p{sexp_i} is a variable \p{s_i}, clause \c{\p{v_i} in \p{s_i}} can
also be written as \p{s_i(v_i)}.
When \p{bexp} is \c{true}, \c{, \p{bexp}} can be omitted.
\begin{code}
    \{\p{exp}: \p{v\sb{1}} in \p{sexp\sb{1}}, ..., \p{v\sb{k}} in \p{sexp\sb{k}}, \p{bexp}\}
\end{code}
To indicate that a variable \c{x} on the left side of a membership
clause is a parameter, add prefix \c{=} to \c{x}; this is only needed
for the first occurrence of such a variable.
Notation \c{=x} means a value that is equal to the value of parameter
\c{x}; it is equivalent to using a fresh variable \c{y} in its place and
adding a conjunct \c{y=x} in condition \p{bexp}.
This notation can generalize: one can add as prefix any binary operator
that is a symbol not allowed in identifiers, uses the parameter value as
the right operand, and returns a Boolean value.  For example, \c{>x} means
a value that is greater than the value of parameter \c{x}.

\langmap
\begin{code}
    setof(\p{exp}, \p{v\sb{1}} in \p{sexp\sb{1}}, ..., \p{v\sb{k}} in \p{sexp\sb{k}}, \p{bexp})
\end{code}
\quad\quad where \c{\_} is used in place of \c{=} to indicate parameters.
This forbids the use of variable names that start with \c{\_} in the query.
Also, only for \p{sexp_i} being variable \c{received} or \c{sent} can
clause \c{\p{v_i} in received} or \c{\p{v_i} in sent} be written as
\c{received(\p{v_i})} or \c{sent(\p{v_i})}, respectively.





\myparag{Aggregations}

An aggregation is a query of one of the following two forms, where
\c{\p{agg}} is an aggregation operator, including \c{count}, \c{sum},
\c{min}, and \c{max}.
Given values of parameters, the query returns the value of applying
\c{\p{agg}} to the set value of \c{\p{sexp}}, for the first form, or
to the multiset of values of \p{exp} for all combinations of values of
variables that satisfy all membership clauses \c{\p{v\sb{i}} in
  \p{sexp\sb{i}}} and condition \p{bexp}, for the second form.
\begin{code}
    \p{agg} \p{sexp}
    \p{agg} (\p{exp}: \p{v\sb{1}} in \p{sexp\sb{1}}, ..., \p{v\sb{k}} in \p{sexp\sb{k}}, \p{bexp})
\end{code}

\langmap
\begin{code}
    \p{agg}(\p{sexp})
    \p{agg}of(\p{exp}, \p{v\sb{1}} in \p{sexp\sb{1}}, ..., \p{v\sb{k}} in \p{sexp\sb{k}}, \p{bexp})
\end{code}
\quad\quad where \c{len} is used in place of \c{count}.


\myparag{Quantifications}

A quantification is a query of one of the following two forms.
The two forms are called existential and universal quantifications,
respectively.
Given values of parameters, the query returns \c{true} iff for some or
all, respectively, combinations of values of variables that satisfy
all membership clauses \c{\p{v\sb{i}} in \p{sexp\sb{i}}}, expression
\p{bexp} evaluates to \c{true}.
When an existential quantification returns \c{true}, all variables in
the query are also bound to a combination of values, called a witness,
that satisfy all the membership clauses and condition \p{bexp}.
\begin{code}
    some \p{v\sb{1}} in \p{sexp\sb{1}}, ..., \p{v\sb{k}} in \p{sexp\sb{k}} has \p{bexp}
    each \p{v\sb{1}} in \p{sexp\sb{1}}, ..., \p{v\sb{k}} in \p{sexp\sb{k}} has \p{bexp}
\end{code}

\langmap\notes{}
\begin{code}
    some(\p{v\sb{1}} in \p{sexp\sb{1}}, ..., \p{v\sb{k}} in \p{sexp\sb{k}}, has = \p{bexp})
    each(\p{v\sb{1}} in \p{sexp\sb{1}}, ..., \p{v\sb{k}} in \p{sexp\sb{k}}, has = \p{bexp})
\end{code}
\quad\quad where prefix \c{\_} or a \c{params} clause is used to indicate
parameters, as for comprehensions.

\notes{}


\myparag{Patterns}
\label{sec-patterns}

In the clauses \c{\p{v\sb{1}} in \p{sexp\sb{1}}, \ldots, \p{v\sb{k}}
  in \p{sexp\sb{k}}} in all of comprehensions, aggregations, and
quantifications, a pattern, in particular a tuple expression
\p{texp_i}, called a tuple pattern, may occur in place of variable
\p{v_i}.  Previously unbound variables in \p{texp_i} are bound to the
corresponding components in %
the matched elements of the value of \p{sexp\sb{i}}.
The underscore (\c{\_}) is used as a wild card that can be bound to anything.
In general, any data construction expression 
can be used as a pattern; we use only tuple patterns because messages are
by convention tuples.


\mysubsec{Configurations}



\myparag{Channel types}

The following statement configures all channels to be
first-in-first-out (FIFO).
Other options for \c{channel} include \c{reliable} and \c{\{reliable,
  fifo\}}.  When these options are specified, TCP is used for process
communication; otherwise, UDP is used.
\begin{code}
    configure channel = fifo
\end{code}

\langmap
\begin{code}
    config(channel = 'fifo')
\end{code}
Channels can also be configured separately for messages from certain types
of processes to certain types of processes, by adding clauses \c{from
  \p{ps}} and \c{to \p{qs}}, or arguments \c{from\_ = \p{ps}} and \c{to =
  \p{qs}} in Python syntax, where \p{ps} and \p{qs} can be a type of
processes or a set of types of processes.  Each of these clauses is
optional; the default is all types of processes.


\myparag{Message handling}

The following statement configures the system to handle all un-handled
messages at each yield point; this is the default.
Other options for \c{handling} include \c{one}.
\begin{code}
    configure handling = all
\end{code}

\langmap
\begin{code}
    config(handling = 'all')
\end{code}


\myparag{Logical clocks}

The following statement configures the system to use Lamport clock.  Other
options for \c{clock} include \c{vector}; it is currently not supported.
\begin{code}
    configure clock = Lamport
\end{code}

\langmap
\begin{code}
    config(clock = 'Lamport')  
\end{code}

A call \c{logical\_time()} returns the current value of the logical
clock.


\myparag{Overall}

A DistAlgo program is written in files named with extension \c{.da}.  It
consists of a set of process definitions, a method \c{main}, and possibly
other, conventional program parts.  Method \c{main} specifies the
configurations and creates, sets up, and starts a set of processes.

DistAlgo language constructs can be used in process definitions and method
\c{main} and are implemented according to the semantics described; 
other, conventional program parts are implemented according to their
conventional semantics.
\notes{}



\mysubsec{Other useful functions in Python}

\myparag{Logging output}

The following method prints the values of expressions \c{\p{exp_1}} through
\c{\p{exp_k}} in their \c{str()} representation, separated by the value of
\c{\p{str\_exp}} and prefixed with system timestamp, process id, and the
specified integer level \p{l}, to the log of %
the node that runs the current DistAlgo process; the printing is done only
if level \p{l} is greater or equal to the default logging level or the
level specified on the command line when starting the node. %
The log defaults to console\notes{}\notes{}, but can be a file specified on the command
line when starting the node. %
\begin{code}
    output(\p{exp\sb{1}}, ..., \p{exp\sb{k}}, sep = \p{str\_exp}, level = \p{l})
\end{code}
Argument \c{sep} is optional and defaults to the empty space.  Argument
\c{level} is optional and defaults to \c{logging.INFO}, corresponding to
value 20, in the Python logging module; see
\url{https://docs.python.org/3/library/logging.html#levels} for a list of
predefined level names.


\myparag{Importing modules}

The following statement is equivalent to Python statement \c{import
  \p{module} as \p{m}}.\linebreak It takes DistAlgo module \p{module},
which must end in a DistAlgo program file name excluding extension \c{.da},
compiles the program file if an up-to-date compiled file does not already
exist, and assigns to \p{m} the resulting module object if successful or
raises \c{ImportError} otherwise.
\begin{code}
    \p{m} = import_da(\p{module})
\end{code}


\end{document}
